\documentclass[aps,prb,twocolumn,showpacs,superscriptaddress,10p,longbibliography]{revtex4-1}
%\bibliographystyle{apsrev}
\usepackage{graphicx}
\usepackage{amssymb}
\usepackage{amsmath}
\usepackage{bm}
\usepackage{xcolor}
\usepackage{subfigure}
\usepackage{float}
\usepackage[colorlinks=true,linkcolor=red, citecolor = magenta]{hyperref}
\usepackage{gensymb}
\usepackage[sort&compress]{natbib}
\usepackage[shortlabels]{enumitem}
\usepackage{scalerel}
\usepackage{dsfont}
%Defined commands
%=====================================================================
\newcommand{\uu}{\uparrow}
\newcommand{\dd}{\downarrow}

\newcommand{\expect}[1]{\langle{#1}\rangle}
\newcommand{\ket}[1]{|{#1}\rangle}
\newcommand{\bra}[1]{\langle{#1}|}
\newcommand{\braket}[2]{\langle{#1}|{#2}\rangle}
\newcommand{\abs}[1]{\left|{#1}\right|}

\newcommand{\rre}{\mathrm{Re}\,} 
\newcommand{\iim}{\mathrm{Im}\,} 

\newcommand{\Tr}{\mathrm{Tr}}

\newcommand{\rr}{\vec{r}}
\newcommand{\AAA}{\vec{A}}

\newcommand{\aalpha}{\boldsymbol{\alpha}}
\newcommand{\bbeta}{\boldsymbol{\beta}}
\newcommand{\ttheta}{\boldsymbol{\theta}}
\newcommand{\pphi}{\boldsymbol{\varphi}}
\newcommand{\ssigma}{\hat{\boldsymbol{\sigma}}}

\newcommand{\Hbdg}{\mathcal{H}}

\begin{document}

\title{Inversion symmetry in the extended BHZ model} \author{J\'anos K. Asb\'oth}
\affiliation{Department of Theoretical Physics and BME-MTA Exotic
  Quantum Phases Research Group, Budapest University of Technology and
  Economics, H-1111 Budapest, Hungary}
\affiliation{Institute for Solid State Physics and
  Optics, Wigner Research Centre, H-1525 Budapest P.O. Box 49,
  Hungary}
\email{janos.asboth@wigner.hu}
\date{\today}

\begin{abstract}
  
\end{abstract}

\maketitle

We first review the standard BdG setting for symmetries, and how
inversion symmetry appears here, and how to transition to a chiral basis.

We then discuss how the BHZ model fits into this framework.

\section{Standard BdG for time-reversal symmetric superconductors (DIII)}

A superconductor in the mean-field approximation can be described by a
single-particle Bogoliubov-de Gennes (BdG) Hamiltonian $\Hbdg$. This
has the form, 
\begin{align}
  \Hbdg &= \begin{pmatrix} 
    h & \Delta \\ 
    -\Delta^\ast & -h^\ast \end{pmatrix}. 
\end{align}
Here $h$ and $\Delta$ are possibly quite large matrices. 
Using $\tau_{x,y,z}$ to denote Pauli matrices acting on the blocks
here, this has particle-hole symmetry by construction,
\begin{align}
  \mathcal{P} &= \tau_x K; &
  \tau_x K \Hbdg K \tau_x &= -\Hbdg,
\end{align}
with $K$ representing complex conjugation in position basis. 

We are interested in superconductors with time-reversal symmetry (that
squares to -1). Time-reversal is usually represented by
\begin{align}
  \mathcal{T} = \sigma_y K,
\end{align}
where $\sigma_{x,y,z}$ act on spin (or in general some internal degree
of freedom). This should act on the BdG Hamiltonian block by block,
i.e., we need
\begin{align}
  [\sigma_y,\tau_x] = 0,
\end{align}
which holds if we have a tensor product structure.

If we have both particle-hole and time-reversal symmetries, their
product, chiral symmetry, is also a symmetry of the system. This is
represented in the standard basis used above by
\begin{align}
  \hat\Gamma &= \tau_x \sigma_y
  = \begin{pmatrix} 0 & 0 & 0 & -i\\
    0 & 0 & i & 0\\
    0 & -i & 0 & 0\\
    i & 0 & 0 & 0
  \end{pmatrix}.
\end{align}
Although this looks like a $4\times 4$ matrix, there could be any
number of internal states, and so each number in the matrix above
can correspond to a unit matrix. 

\subsection{Chiral basis}

We are interested in transforming the Hamiltonian to the chiral basis, where
chiral symmetry is represented by 
\begin{align}
  \tilde{\Gamma} = \tau_z.
\end{align}
To achieve this transformation, we diagonalize $\hat{\Gamma}$.
\begin{align}
  \hat{\Gamma} &= ||| \, \diagdown \equiv = \mathcal{O} \tau_z \mathcal{O}^\dagger;\\
  \mathcal{O} &= \frac{1}{\sqrt 2}
  \begin{pmatrix} 1 & 0 & 1 & 0\\
    0 & 1 & 0 & 1 \\
    0 & -i & 0 & i\\
    i & 0 & -i & 0
  \end{pmatrix} =
  \frac{1}{\sqrt 2}
  \begin{pmatrix}
    1 & 1 \\ \sigma_y & -\sigma_y.
    \end{pmatrix}
\end{align}
The transformation works by conjugating with $\mathcal{O}$ above,
which gives for the particle-hole and time-reversal symmetries,
\begin{align}
  \tau_x K &\to \frac{1}{2}
  \begin{pmatrix}
    1 & \sigma_y \\ 1 & -\sigma_y
  \end{pmatrix}
  \begin{pmatrix}  0 & 1 \\ 1 & 0 \end{pmatrix} K  
  \begin{pmatrix}
    1 & 1 \\ \sigma_y & -\sigma_y
  \end{pmatrix} \nonumber \\
  = \frac{1}{2}
  \begin{pmatrix}
    1 & \sigma_y \\ 1 & -\sigma_y
  \end{pmatrix}&
  \begin{pmatrix}
   -\sigma_y & \sigma_y \\ 1 & 1
  \end{pmatrix}
  K =
  \begin{pmatrix}
   0 & \sigma_y \\ -\sigma_y &0
  \end{pmatrix} K
  =
  i \tau_y \sigma_y K,
\end{align}
where we can drop the factor $i$ at the end for simplicity.
Similarly, for time reversal,  
\begin{align}
  \sigma_y K &\to \tau_x \sigma_y K.
\end{align}

In the chiral basis, the BdG Hamiltonian is block off-diagonal.
\begin{align}
  \Hbdg &= \begin{pmatrix} 0 & D \\ D^\dagger & 0 \end{pmatrix}.
\end{align}
The particle-hole symmetry acts on $D$ by:
\begin{align}
  D \to -\sigma_y D^T \sigma_y,
\end{align}
as can be checked by calculation. 
For the bulk momentum-space Hamiltonian, the requirement of particle-hole
symmetry, $\Hbdg = - \mathcal{P}\Hbdg\mathcal{P}^{-1}$ thus translates to
\begin{align}
  D(k) &= \sigma_y D(-k)^T \sigma_y.
\end{align}
As a consequence, in this basis,
\begin{align}
 \label{eq:detD1}
  \det D(k) = \det D(-k).
\end{align}

A consequence of the above is that winding number of $\det D(k)$
around some momentum point $k_0$ is the same as around its
time-reversed partner, $-k_0$. Thus topologically protected Weyl nodes
come in pairs, where members of a pair have equal charges. Because the
net sum of all charges in the Brillouin zone has to be 0, Weyl nodes
should come in groups of 4. This was Benjamin Beri's statement.

\subsection{Inversion symmetry}

Inversion is an operation that acts in real space by inversion through
a center. In momentum space this changes $k \leftrightarrow -k$, but
it can also change the internal states by some unitary $\hat J$.

We expect inversion to not affect the particle/hole degree of
freedom. Not sure about spin, probably should not affect spin
either. So it should act elementwise in the chiral basis as
well. Thus inversion symmetry gives
\begin{align}
  D(k) = \hat{J} D(-k) \hat{J}^\dagger.
\end{align}
This requires
\begin{align}
  \det D(k) = \det D(-k),
\end{align}
which is the same requirement as that due to particle-hole symmetry,
Eq.~\eqref{eq:detD1}.


\section{Our case, in special basis}

We are investigating the BHZ model, where we interpret the Hamiltonian
as a BdG Hamiltonian. Here the bulk momentum-space Hamiltonian reads,
\begin{align}
  \label{eq:coupled_BHZ}
  H(k) &=
  \begin{pmatrix}
  A(k) & B(k) & 0 & -C \\
  B(k)^\ast & -A(k) & C & 0 \\
  0 & C & A(k) & -B(k)^\ast \\
  -C & 0 & -B(k) & A(k),
  \end{pmatrix}
\end{align}
with
\begin{align}
  A(k) &= u + v_x \cos k_x + \cos k_y;\\
  B(k) &= v_x \sin k_x - i\sin k_y,
\end{align}
and with $()^\ast$ denoting elementwise complex conjugation (without
changing sign of momentum).  This is just two layers of QWZ model,
with time-reversal symmetry built into the model.
The coupling between the layers is such that time reversal
symmetry, represented by
\begin{align}
  \mathcal{T} &= \tau_y K,
\end{align}
is respected.
We happen to also have particle-hole symmetry, with 
\begin{align}
  \mathcal{T} &= \sigma_x K.
\end{align}

We can rewrite the coupled BHZ model to the ``standard basis'' above, and find
\begin{align}
  \Hbdg(k) &=
  \begin{pmatrix}
  A(k) & 0 & B(k) & -C \\
  0 & A(k) & C & -B(k)^\ast \\
  B(k)^\ast & C & -A(k) & 0 \\
  -C & -B(k) & 0 & -A(k),
  \end{pmatrix}
\end{align}
Seen as a BdG Hamiltonian, this is somewhat peculiar. The Hamiltonian
$h(k)$ is just a nn hopping on a square lattice, spin independent.
The superconducting order parameter is momentum dependent, but in a
somewhat weird way.

\subsection{Our inversion symmetry}

The BHZ model, without the coupling, has an inversion
symmetry that commutes with time reversal. It is represented by
\begin{align}
  \hat{J} &= \hat{\sigma}_z.
\end{align}

When interpreting the $\sigma$ matrices as particle-hole operators,
all we do is replace $\sigma \leftrightarrow \tau$, the inversion
becomes peculiar, it is $\hat{J} =\tau_z$. In the language of the BdG
Hamiltonian, it requires
\begin{align}
  h(k) &= h(-k);&   \Delta(k) &= -\Delta(-k). 
\end{align}
The coupling via $C$ breaks inversion symmetry.

In the chiral basis, this inversion symmetry of ours is transformed to:
\begin{align}
\hat{J} \to \tau_x.
\end{align}
Therefore, this inversion symmetry requires
\begin{align}
  D(k) &= D^\dagger(-k),
\end{align}
which entails
\begin{align}
  \det D(k) &= \det D(-k)^\ast.
\end{align}
Together with particle-hole symmetry, Eq.~\eqref{eq:detD1}, this
amounts to
\begin{align}
  \det D(k) \in \mathbb{R}. 
\end{align}

Thus our inversion symmetry excludes the presence of topologically
protected Weyl nodes, because it prevents $\det D(k)$ from having a
phase that could wind. 

\section{Questions}

Our inversion is pretty specific. What property does it have that
prevents protected Weyl nodes?


\end{document}
